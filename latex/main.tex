%% De klasse van het document.
\documentclass[a4paper]{article}

%% De encoding van het document.
\usepackage[utf8]{inputenc}
\usepackage[T1]{fontenc}

%% De gegevens.
\title{Pollution Detection System\\User Manual}
\author{PLA25}
\date{January 5th, 2018}

%% Het document zelf.
\begin{document}

\clearpage
\maketitle
\vspace*{\fill}
%% De groepsleden.
\paragraph{Group members}
~\\\\
\begin{tabbing}
Brian Karmelk \` leerlingnummer
\\
Joey Blankendaal \` leerlingnummer
\\
Martijn Vegter \` leerlingnummer
\\
Matthijs Snijders \` leerlingnummer
\\
Rico Snoek \` leerlingnummer
\\
Thom de Jong \` 500778147
\end{tabbing}
\thispagestyle{empty}
\setcounter{page}{0}
\pagebreak
\tableofcontents
\pagebreak

\section{introduction}
~\\

\subsection{Who are we?}
We are a group of students who go by the name of PLA25. We visit the HvA, in Amsterdam.
~\\

\subsection{Our product}
The Pollution Detection System is a web-based application that provides information about the pollution on earth. Both light pollution and air pollution are being monitored using this system.
\\
\indent
The system uses sensors that can be placed in a specific area to measure the pollution. These measurements can be monitored using the PDS website. This can be done by using the website's main feature, the map.

\pagebreak

\section{The website and how it works}
~\\

\subsection{Login and the home page}
The login page is simple, use your username and password to login to the website. This will redirect you to the homepage. On the top of this website there is a navigation bar, which is used to visit the several available pages.
\\
\indent
Both the login page and home page contain buttons to switch between the languages Dutch and English.
\\
\indent
(Wat er dan ook op de hoofdpagina moet staan.) blablablablablablablabla bla bla.
~\\

\subsection{The map page}
This features a map which can show you all kinds of data regarding pollution. On the left side there is a sidebar. This sidebar contains many options that can be used, for example: filters with which you can enable the heatmap, light pollution map and more.
\\
\indent
There is also a search bar. With the search bar certain places like cities and towns can be found. Also, it can find a specific location by entering coordinates into the search bar.
~\\

\subsection{The admin page}
(Geen idee wat er hier precies moet komen.)


\end{document}