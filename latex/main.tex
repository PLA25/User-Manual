%% De klasse van het document.
\documentclass[a4paper]{article}

%% De encoding van het document.
\usepackage[utf8]{inputenc}
\usepackage[T1]{fontenc}

%% De gegevens.
\title{Pollution Detection System\\User Manual}
\author{PLA25}
\date{January 5th, 2018}

%% Het document zelf.
\begin{document}

\clearpage
\maketitle
\vspace*{\fill}
%% De groepsleden.
\paragraph{Group members}
~\\\\
\begin{tabbing}
Brian Karmelk \` leerlingnummer
\\
Joey Blankendaal \` leerlingnummer
\\
Martijn Vegter \` leerlingnummer
\\
Matthijs Snijders \` leerlingnummer
\\
Rico Snoek \` leerlingnummer
\\
Thom de Jong \` 500778147
\end{tabbing}
\thispagestyle{empty}
\setcounter{page}{0}
\pagebreak
\tableofcontents
\pagebreak

\section{introduction}
~\\

\subsection{Who are we?}
We are a group of students who go by the name of PLA25. We visit the HvA, in Amsterdam.
~\\

\subsection{Our product}
The Pollution Detection System is a web-based application that provides information about the pollution on earth. Both light pollution and air pollution are being monitored using this system.
\\
\indent
The system uses so called "SensorHubs". SensorHubs are a combination of multiple sensors inside of a box. These SensorHubs can be placed in a specific area to measure the pollution. These measurements can be monitored using the PDS website. This can be done by using the website's main feature, the map.

\pagebreak

\section{The website and how it works}
~\\

\subsection{Login and the home page}
The login page is simple, use your username and password to login to the website. This will redirect you to the homepage.
\\
\indent
On the top of this website there is a navigation bar, which is used to visit the several available pages.
If you want to log out, simply click the logout button at the very end of the navigation bar.
\\
\indent
Both the login page and home page contain buttons to switch between the languages Dutch and English.
\\
\indent
(Wat er dan ook op de hoofdpagina moet staan.) blablablablablablablabla bla bla.
~\\

\subsection{The map page}
This features a map which can show you all kinds of data regarding pollution. On the left side there is a sidebar. This sidebar contains many options that can be used, for example: filters with which you can enable the heatmap, light pollution map and more.
\\
\indent
There is also a search bar. With the search bar certain places like cities and towns can be found. Also, it can find a specific location by entering coordinates into the search bar.
~\\

\subsubsection{The heatmap}
The heatmap shows a representation of the temperature, using colors scaling from bright pink (cold) to red (warm).
~\\

\subsubsection{Andere soorten}
De andere soorten kaarten om weer te geven.

\pagebreak

\subsection{The admin page}
This page is only available to admins, that is why it is called the admin page after all.
\\
\indent
This page shows two tables. One table for the Users and one for the SensorHubs. With admin rights the user can change the data inside of these tables.
\\
\indent
For example, the Users table contains data of all the users. With the add button you can add a user and with the delete button behind a specific user you can delete it. Then there is the edit button. The edit button is used to edit the profile of a specific user.
\\
\indent
The SensorHub table simply shows the location of the used SensorHubs with pinpoints.

\end{document}